\documentclass[12pt,a4paper,sans,fontset=windows]{moderncv} % Font sizes: 10, 11, or 12; paper sizes: a4paper, letterpaper, a5paper, legalpaper, executivepaper or landscape; font families: sans or roman

\usepackage[utf8]{inputenc}
\usepackage[export]{adjustbox}
\usepackage{amssymb,amsmath,amsthm}
\usepackage{times} %字体设置
\usepackage{fontspec}
\setmainfont{Times New Roman}
\usepackage{graphicx}
\usepackage{wrapfig}
\usepackage{multicol}
\usepackage{comment} 
\usepackage{enumitem}
\moderncvstyle{casual} % CV theme - options include: 'casual' (default), 'classic', 'oldstyle' and 'banking'
\moderncvcolor{blue} % CV color - options include: 'blue' (default), 'orange', 'green', 'red', 'purple', 'grey' and 'black'

\usepackage{color}
\usepackage{lipsum} % Used for inserting dummy 'Lorem ipsum' text into the template

\usepackage[scale=0.8]{geometry} % Reduce document margins
\setlength{\hintscolumnwidth}{90pt} % Uncomment to change the width of the dates column
%\setlength{\makecvtitlenamewidth}{10cm} % For the 'classic' style, uncomment to adjust the width of the space allocated to your name

%----------------------------------------------------------------------------------------
%	NAME AND CONTACT INFORMATION SECTION
%----------------------------------------------------------------------------------------

\firstname{Yu} % Your first name
\familyname{Zhang} % Your last name
\title{CV}
%\address{14, Gonghangdae-ro 57gil, Gangseo-gu, Seoul, Republic of Korea}
%\mobile{+82 10 8635 3011}
%\phone{(000) 111 1112}
%\fax{(000) 111 1113}
%\email{jayeon0724@gmail.com}
%\homepage{staff.org.edu/~jsmith}{staff.org.edu/$\sim$jsmith} % The first argument is the url for the clickable link, the second argument is the url displayed in the template - this allows special characters to be displayed such as the tilde in this example
%\extrainfo{additional information}
%\photo[70pt][0.4pt]{pictures/picture} % The first bracket is the picture height, the second is the thickness of the frame around the picture (0pt for no frame)
%\quote{}

%----------------------------------------------------------------------------------------

\begin{document}

\textit{\Huge{\textcolor{gray}{Yu Zhang}}}

\hrulefill


\section{Personal Information}

\vspace{3pt} 

\cvitem{Email}{zhangyumath@tju.edu.cn}

\vspace{3pt} 

\cvitem{Homepage}{https://yuzhangmath.github.io/}

\vspace{3pt}  

\cvitem{Address}{Center for Applied Mathematics,
Tianjin University,
No.92 Weijin Road, \newline
Tianjin 300072, P.R.China}



%\includegraphics[scale=0.04]{yu}

%\begin{figure*}
%\includegraphics[scale=0.04]{yu}
%\end{figure*}


%\cvitem{Email}{zhang.4841@buckeyemail.osu.edu}

%\vspace{2pt} 

%\cvitem{Homepage}{https://yuzhangmath.github.io/}

%\vspace{2pt}  

%\cvitem{Address}{Department of Mathematics, Nankai University, No.94 Weijin Road, Tianjin 300071, P.R.China}



\section{Employment}

\vspace{3pt} 

\cvitem{2023.09 -- \hspace{0.1pt} present}{\textbf{Assistant Professor} Tianjin University, \textit{Tianjin, China}  }

\vspace{3pt} 

\cvitem{2020.09 -- 2023.09}{\textbf{Postdoctoral Researcher} Nankai University, \textit{Tianjin, China}  }



\section{Education}

\vspace{3pt} 

\cvitem{2014.08 -- 2020.05}{\textbf{The Ohio State University} \textit{Columbus, Ohio, USA} 
\newline Ph.D. in Mathematics, Dissertation Advisor: John E. Harper}

\vspace{3pt} 

\cvitem{2010.09 -- 2014.07}{\textbf{Peking University} \textit{Beijing, China}
\newline B.S. in Mathematics, Dissertation Advisor: Huijun Fan}

%\vspace{2pt} 

%\cvitem{2007.09 -- 2010.06}{\textbf{TangShan No.1 High School} \textit{Tangshan, Hebei, China}
%}


\section{Grants}

\vspace{3pt} 

\cvitem{2025.01 -- 2027.12}{National Natural Science Foundation of China (No. 12401085): Research on odd primary components of the stable homotopy groups of spheres based on motivic homotopy theory and computer-assisted calculations; \textbf{Principal Investigator}.}

\vspace{3pt} 

\cvitem{2024.01 -- 2025.12}{Tianjin University Technology Innovation Leading Talent Cultivation Plan Qiming Program (No. 2024XQM-0009): Computing the stable homotopy groups of spheres; \textbf{Principal Investigator}.}

\vspace{3pt} 

\cvitem{2023.01 -- 2026.12}{National Natural Science Foundation of China (No. 12261091): Some problems on the spectral sequences and the cohomology of the classifying space $BPU_n$; \textbf{Co-Principal Investigator}.}

\vspace{3pt} 

\cvitem{2023.01 -- 2026.12}{National Natural Science Foundation of China (No. 12271183): Cohomology of toric spaces and motivic stable homotopy; \textbf{Co-Investigator}.}

\vspace{3pt} 

\cvitem{2021.05 -- 2023.04}{Project on Postdoctoral International Exchanges: Properties of nilpotent structured ring spectra. \textbf{Principal Investigator}.}

\vspace{3pt} 

\cvitem{2019.01 -- 2022.12}{National Natural Science Foundation of China (No. 11871284): Researches on Motivic stable homotopy and R-S spectral sequence in Toric topology; \textbf{Co-Investigator}.}

\vspace{10pt} 

\section{Research Papers}

\vspace{3pt} 

\begin{itemize}[wide=30pt, leftmargin=*]
    \item \textbf{The secondary periodic element $\beta_{p^2/p^2-1}$ and its applications}
        \newline (joint with J. Hong and X. Wang), \textit{SCIENCE CHINA Mathematics}. 68, 207-222 (2025).
    \item \textbf{Detecting nontrivial products in the stable homotopy ring of spheres via the third Morava stabilizer algebra}
    \newline (joint with X. Wang, J. Wu, and L. Zhong), \textit{Proceedings of the American Mathematical Society}. 152 (2024), 4521-4536.
    \item \textbf{A correspondence between higher Adams differentials and higher algebraic Novikov differentials at odd primes}
        \newline (joint with X. Wang),  \textit{Proceedings of the American Mathematical Society}. 151 (2023), 5087-5096. 
    \item \textbf{The $p$-primary subgroup of the cohomology of $BPU_n$ in dimension $2p+6$}
        \newline (joint with Z. Zhang and L. Zhong), \textit{Topology and Its Applications}.  338 (2023): 108642.
    \item \textbf{Some nontrivial secondary Adams differentials on the fourth line}
        \newline (joint with X. Wang and Y. Wang),  \textit{New York Journal of Mathematics}.  29 (2023) 687-707.
    \item \textbf{Homotopy pro-nilpotent structured ring spectra and topological Quillen localization}
        \newline \textit{Journal of Homotopy and Related Structures}. 17(4), 511-523 (2022).
    \item \textbf{The $p$-primary subgroups of the cohomology of $BPU_n$ in dimensions less than $2p+5$}
        \newline (joint with X. Gu, Z. Zhang, and L. Zhong), \textit{Proceedings of the American Mathematical Society}. 150(9), 4099-4111 (2022). 
    \item \textbf{Topological Quillen localization of structured ring spectra}
        \newline (joint with J. E. Harper), \textit{Tbilisi Mathematical Journal}. 12(3), 69-91 (2019). 
\end{itemize}

\vspace{2pt} 




\section{Invited talks}

\vspace{2pt} 

\cvitem{Mar 13, 2025}{\textbf{Tsinghua Sanya International Mathematics Forum}, Stable homotopy groups of spheres: Theories and Computations, where to now?, "Computing the Adams $E_2$-Page for Odd Primes"}

\vspace{2pt} 

\cvitem{Oct 18, 2024}{\textbf{Yanbian University}, Topology seminar (online), "The convergence of $h_0h_3$ in the odd primary Adams spectral sequence"}

\vspace{2pt} 

\cvitem{Dec 06, 2023}{\textbf{Peking University}, Geometry and Topology seminar, "The algebraic Novikov spectral sequence and stable homotopy groups of spheres at odd primes"}


\cvitem{May 14, 2023}{\textbf{Nankai University}, 2023 Topology and its Applications Conference, "Motivic method and stable homotopy groups of spheres"}

\vspace{2pt} 

\cvitem{Nov 19, 2022}{\textbf{Tianjin University of Technology}, Annual Meeting of Math and Statistics in Tianjin, "Stable homotopy groups of the sphere at odd primes"}

\vspace{2pt} 

\cvitem{Aug 10, 2022}{\textbf{Inner Mongolia Minzu University}, National Conference for Young Topologists, "Some Differentials in the Algebraic Novikov Spectral Sequence and Its Applications"}

\vspace{2pt} 

\cvitem{Nov 23, 2021}{\textbf{Chern Institute of Mathematics}, Higher Structures in Geometry and Physics (online), "Structured ring spectra and Topological Quillen homology"}

\vspace{2pt} 

\cvitem{Jul 25, 2021}{\textbf{Shanghai Center for Mathematical Sciences}, IWoAT 2021 Junior Researcher Forum, "Some spectral sequence computations towards stable homotopy groups of spheres at odd primes"}

\vspace{2pt} 

\cvitem{Jul 17, 2021}{\textbf{Northeast Normal University}, National Conference for Young Topologists, "Some spectral sequence computations towards stable homotopy groups of spheres at odd primes"}

\vspace{2pt} 

\cvitem{Dec 16, 2020}{\textbf{Vietnam National University}, The 8th East Asian Conference on Algebraic Topology (online), "Koszul duality and TQ-homological Whitehead theorem of structured ring spectra"}

\vspace{2pt} 

\cvitem{Nov 4, 2020}{\textbf{University of Regina}, Topology Seminar (online), "Koszul duality and TQ-homological Whitehead theorem of structured ring spectra"}

\vspace{2pt} 

\cvitem{Nov 21, 2019}{\textbf{Binghamton University}, Geometry and Topology Seminar, "Topological Quillen localization of structured ring spectra"}

\vspace{2pt} 

\cvitem{Nov 19, 2019}{\textbf{The University of Chicago}, Algebraic Topology Seminar, "Topological Quillen localization of structured ring spectra"}

\vspace{2pt} 

\cvitem{Nov 1, 2019}{\textbf{Penn State Altoona}, Topology Seminar, "Topological Quillen localization of structured ring spectra"}

\vspace{2pt} 

\cvitem{Oct 30, 2019}{\textbf{Indiana University Bloomington}, Topology Seminar, "Topological Quillen localization of structured ring spectra"}

\vspace{2pt} 

\cvitem{Oct 29, 2019}{\textbf{University of Illinois at Urbana-Champaign}, Topology Seminar, "Topological Quillen localization of structured ring spectra"}

\vspace{2pt} 

\cvitem{Jul 29, 2019}{\textbf{Norwegian University of Science and Technology}, Equivariant Topology and Derived Algebra conference, "Topological Quillen completion and localization of structured ring spectra"}

\vspace{2pt} 

\cvitem{Mar 30, 2019}{\textbf{University of Illinois at Urbana-Champaign}, Graduate Student Topology and Geometry Conference, "An easy proof of the homological Whitehead theorem for nilpotent spaces"}

\vspace{2pt} 

\cvitem{Sep 27, 2018}{\textbf{The Ohio State University}, Homotopy Theory Seminar, "Localization of structured ring spectra with respect to TQ homology"}

\vspace{2pt}

\cvitem{Jul 10, 2018}{\textbf{University of Copenhagen}, Young Topologists Meeting, "Bousfield localization of structured ring spectra"}

\vspace{2pt}


\begin{comment}



\section{Informal talks}

\vspace{2pt} 

\cvitem{Mar 2021}{\textbf{Nankai University}, Topology Seminar, "Introduction to motivic homotopy theory 1"}

\vspace{2pt} 

\cvitem{Dec 2020}{\textbf{Nankai University}, Topology Seminar, "Homology is abelianization of homotopy"}

\vspace{2pt} 

\cvitem{Nov 2020}{\textbf{Nankai University}, Topology Seminar, "Spectral sequence for filtered chain complex"}

\vspace{2pt} 

\cvitem{Nov 2020}{\textbf{Nankai University}, Topology Seminar, "Operads and structured ring spectra"}

\vspace{2pt} 

\cvitem{Oct 2020}{\textbf{Nankai University}, Topology Seminar, "Bousfield localization of spectra"}

\vspace{2pt} 

\cvitem{Oct 2020}{\textbf{Nankai University}, Topology Seminar, "Bousfield completion and Adams tower"}

\vspace{2pt} 

\cvitem{Sep 2020}{\textbf{Nankai University}, Topology Seminar, "Stable homotopy groups, Bousfield-Kan completion and Bousfield localization"}


\vspace{2pt} 

\cvitem{Sep 2019}{\textbf{The Ohio State University}, Student Homotopy Theory Seminar, "What can be seen by topological Quillen homology?"}

\vspace{2pt}

\cvitem{Mar 2019}{\textbf{The Ohio State University}, Student Homotopy Theory Seminar, "Induce model structure along adjunctions"}

\vspace{2pt}

\cvitem{Feb 2019}{\textbf{The Ohio State University}, Student Homotopy Theory Seminar, "Reedy model structure"}

\vspace{2pt}

\cvitem{Sep 2018}{\textbf{The Ohio State University}, Student Homotopy Theory Seminar, "Postnikov tower and obstruction theory for ring spectra"}

\vspace{2pt}


\cvitem{Sep 2018}{\textbf{The Ohio State University}, Student Homotopy Theory Seminar, "Postnikov tower and obstruction theory for spaces"}

\vspace{2pt}


\cvitem{Sep 2018}{\textbf{The Ohio State University}, Graduate Student Seminar, "Localization methods in algebraic topology"}

\vspace{2pt}

\cvitem{Mar 2018}{\textbf{The Ohio State University}, Student Homotopy Theory Seminar, "Constructing Bousfield localization of spaces"}

\vspace{2pt}

\cvitem{Sep 2017}{\textbf{The Ohio State University}, Student Homotopy Theory Seminar, "Bousfield localization of model category"}

\vspace{2pt}

\cvitem{May 2017}{\textbf{Gooding Talbot house}, The MIT Talbot Workshop: Obstruction theory for structured ring spectra, "Topological Andre-Quillen cohomology"}

\vspace{2pt}

\cvitem{Feb 2017}{\textbf{The Ohio State University}, Student Infinity Category Seminar, "Spectra and Stable Homotopy Category"}

\vspace{2pt}

\end{comment}


\section{Conference organization}

\vspace{3pt} 

\cvitem{May 2023}{\textbf{2023 Topology and its Applications Conference}, Nankai University, Co-organizer}





\section{Conference participation}

\vspace{3pt} 

\cvitem{Mar 2025}{\textbf{Stable homotopy groups of spheres: Theories and Computations, where to now?}, at Tsinghua Sanya International Mathematics Forum}

\vspace{2pt} 

\cvitem{Mar 2025}{\textbf{Special international conference in honor of Prof. Peter May}, at Southern University of Science and Technology}

\vspace{2pt} 

\cvitem{Sep 2024}{\textbf{Computer assisted studies in Geometry and Topology}, at Tianyuan Mathematics Research Center}

\vspace{2pt} 

\cvitem{Jul 2024}{\textbf{International conference at Fudan University}, at Shanghai Center for Mathematical Sciences, Fudan University}

\vspace{2pt} 

\cvitem{Jun 2024}{\textbf{IWoAT 2024 summer school on motivic homotopy theory}, at Shanghai Center for Mathematical Sciences, Fudan University}

\vspace{2pt} 

\cvitem{Dec 2023}{\textbf{International Workshop on Algebraic Topology 2023 Winter School}, at Southern University of Science and Technology}

\vspace{2pt} 

\cvitem{Aug 2023}{\textbf{IWoAT Summer School 2023: Operads, spectra, and multiplicative structures}, at Yanqi Lake Beijing Institute of Mathematical Sciences and Applications}

\vspace{2pt} 

\cvitem{Jul 2023}{\textbf{International Workshop on Algebraic Topology 2023}, at Beijing International Center for Mathematical Research}

\vspace{2pt} 

\cvitem{May 2023}{\textbf{2023 Topology and its Applications Conference}, at Nankai University}

\vspace{2pt} 

\cvitem{Nov 2022}{\textbf{Annual Meeting of Math and Statistics in Tianjin}, organized jointly by Tianjin University of Technology and Nankai University}

\vspace{2pt}

\cvitem{Aug 2022}{\textbf{Summer School on Chromatic Homotopy Theory and Higher (Infinity-Categorical) Algebra (online)}, organized by Shanghai Center for Mathematical Sciences, Fudan University, and Yanqi Lake Beijing Institute of Mathematical Sciences and Applications}

\vspace{2pt}

\cvitem{Aug 2022}{\textbf{National Conference for Young Topologists}, at Inner Mongolia Minzu University}

\vspace{2pt}

\cvitem{Jul 2022}{\textbf{Mid-South Algebraic Topology and Geometry Workshop (online)}, organized by Huazhong University of Science and Technology and South China University of Technology}

\vspace{2pt}

\cvitem{Nov 2021}{\textbf{Higher Structures in Geometry and Physics (online)}, organized by Chern Institute of Mathematics}

\vspace{2pt}

\cvitem{Jul 2021}{\textbf{IWoAT 2021 Junior Researcher Forum (online)}, organized by Shanghai Center for Mathematical Sciences}

\vspace{2pt}

\cvitem{Jul 2021}{\textbf{Summer School on Equivariant Homotopy Theory}, at Shanghai Center for Mathematical Sciences}

\vspace{2pt}

\cvitem{Jul 2021}{\textbf{National Conference for Young Topologists}, at Northeast Normal University}

\vspace{2pt}

\cvitem{Dec 2020}{\textbf{The 8th East Asian Conference on Algebraic Topology (online)}, organized jointly by Vietnam National University and Vietnam Institute for Advanced Study in Mathematics}

\vspace{2pt}

\cvitem{Aug 2020}{\textbf{The First Joint Workshop on Topology and Data Science}, at Chongqing University of Technology}

\vspace{2pt}

\cvitem{May 2020}{\textbf{Midwest Topology Seminar (online)}, organized by Wayne State University}

\vspace{2pt}

\cvitem{Oct 2019}{\textbf{Mayday 2019}, at The University of Chicago}

\vspace{2pt}

\cvitem{Sep 2019}{\textbf{AMS Fall Central Sectional Meeting}, at University of Wisconsin-Madison}

\vspace{2pt}


\cvitem{Aug 2019}{\textbf{International Workshop on Algebraic Topology}, at Shanghai Center for Mathematical Sciences}

\vspace{2pt}


\cvitem{Aug 2019}{\textbf{Summer School on Equivariant Homotopy Theory}, at Shanghai Center for Mathematical Sciences}

\vspace{2pt}


\cvitem{Jul 2019}{\textbf{Equivariant Topology \& Derived Algebra}, at Norwegian University of Science and Technology}

\vspace{2pt}


\cvitem{Jul 2019}{\textbf{Young Topologists Meeting}, at École polytechnique fédérale de Lausanne}

\vspace{2pt}


\cvitem{May 2019}{\textbf{Midwest Topology Seminar}, at Michigan State University}

\vspace{2pt}


\cvitem{Apr 2019}{\textbf{Shanks Workshop on Homotopy Theory}, at Vanderbilt University}

\vspace{2pt}


\cvitem{Mar 2019}{\textbf{Graduate Student Topology and Geometry Conference}, at University of Illinois at Urbana-Champaign}

\vspace{2pt}

\cvitem{Mar 2019}{\textbf{Workshop on Functor Calculus}, at The Ohio State University}

\vspace{2pt}


\cvitem{Sep 2018}{\textbf{Midwest Topology Seminar}, at University of Kentucky}

\vspace{2pt}

\cvitem{Jul 2018}{\textbf{Young Topologists Meeting}, at University of Copenhagen}

\vspace{2pt}


\cvitem{Apr 2018}{\textbf{Midwest Topology Seminar}, at Indiana University Bloomington}

\vspace{2pt}


\cvitem{Mar 2018}{\textbf{AMS Special Session on Homotopy Theory}, at The Ohio State University}

\vspace{2pt}


\cvitem{Mar 2018}{\textbf{Midwest Topology Seminar}, at Northwestern University}

\vspace{2pt}

\cvitem{Nov 2017}{\textbf{Midwest Topology Seminar}, at Wayne State University}

\vspace{2pt}

\cvitem{Jul 2017}{\textbf{Homotopy Theory: Tools and Applications}, at University of Illinois at Urbana-Champaign}

\vspace{2pt}

\cvitem{May 2017}{\textbf{The MIT Talbot Workshop: Obstruction theory for structured ring spectra}, at Gooding Talbot house}

\vspace{2pt}

\cvitem{Mar 2017}{\textbf{Shanks Workshop on Homotopy Theory}, at Vanderbilt University}

\vspace{2pt}

\cvitem{Feb 2017}{\textbf{Conference for Young researchers in homotopy theory and categorical structures}, at Max Planck Institute for Mathematics}

\vspace{2pt}



\section{Teaching Experience}

\vspace{2pt}
 
\cvitem{}{As lecturer in Tianjin University}

\vspace{2pt}

\cvitem{Autumn 2024}{\textbf{Calculus I}
\newline Student Evaluation: 97.6/100}

\vspace{4pt}
 
\cvitem{}{As lecturer in Nankai University}
 
\vspace{2pt}

\cvitem{Spring 2022}{\textbf{Probability and Statistics for Liberal Arts}
\newline Student Evaluation: 97.7/100}

\vspace{2pt}

\cvitem{Autumn 2021}{\textbf{Advanced Mathematics for Liberal Arts}
\newline Student Evaluation: 99.0/100}

\vspace{4pt}

\cvitem{}{As teaching associate at The Ohio State University}

\vspace{2pt}

\cvitem{Autumn 2018}{\textbf{MATH 2177: Mathematical Topics for Engineers}}

\vspace{2pt}

\cvitem{Autumn 2017}{\textbf{MATH 1172: Engineering Mathematics A}}

\vspace{2pt}

\cvitem{Autumn 2016}{\textbf{MATH 1172: Engineering Mathematics A}}

\vspace{2pt}

\cvitem{Spring 2016}{\textbf{MATH 1152: Calculus II}}

\vspace{2pt}



\section{Languages}

\vspace{2pt}

\cvitem{}{\textbf{English}: fluent (speaking, reading, writing)}

\vspace{2pt}

\cvitem{}{\textbf{French}: basic (reading)}

\vspace{2pt}

\cvitem{}{\textbf{Mandarin}: native language}



\end{document}