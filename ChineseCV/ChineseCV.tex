\documentclass[12pt,a4paper,sans,fontset=windows]{moderncv} % Font sizes: 10, 11, or 12; paper sizes: a4paper, letterpaper, a5paper, legalpaper, executivepaper or landscape; font families: sans or roman
%\usepackage{xeCJK}
%\usepackage[utf8]{inputenc}
%\usepackage[export]{adjustbox}
\usepackage{amssymb,amsmath,amsthm}
\usepackage{times} %字体设置
\usepackage{fontspec}
%\setmainfont{Times New Roman}
\usepackage{graphicx}
\usepackage{wrapfig}
\usepackage{multicol}
\usepackage{enumitem}
%\usepackage{ctex}

\usepackage{xeCJK}
%\setCJKmainfont{SimSun}  % Windows 上的宋体
\setCJKmainfont{STSong}  % macOS 上的宋体
%中文需要用xelatex编译

\moderncvstyle{casual} % CV theme - options include: 'casual' (default), 'classic', 'oldstyle' and 'banking'
\moderncvcolor{blue} % CV color - options include: 'blue' (default), 'orange', 'green', 'red', 'purple', 'grey' and 'black'
\usepackage{color}
%\usepackage{lipsum} % Used for inserting dummy 'Lorem ipsum' text into the template

\usepackage[scale=0.8]{geometry} % Reduce document margins
\setlength{\hintscolumnwidth}{90pt} % Uncomment to change the width of the dates column 3cm 95pt
%\setlength{\makecvtitlenamewidth}{10cm} % For the 'classic' style, uncomment to adjust the width of the space allocated to your name

%----------------------------------------------------------------------------------------
%	NAME AND CONTACT INFORMATION SECTION
%----------------------------------------------------------------------------------------

\firstname{Yu} % Your first name
\familyname{Zhang} % Your last name

% All information in this block is optional, comment out any lines you don't need
\title{CV}
%\address{14, Gonghangdae-ro 57gil, Gangseo-gu, Seoul, Republic of Korea}
%\mobile{+82 10 8635 3011}
%\phone{(000) 111 1112}
%\fax{(000) 111 1113}
%\email{jayeon0724@gmail.com}
%\homepage{staff.org.edu/~jsmith}{staff.org.edu/$\sim$jsmith} % The first argument is the url for the clickable link, the second argument is the url displayed in the template - this allows special characters to be displayed such as the tilde in this example
%\extrainfo{additional information}
%\photo[70pt][0.4pt]{pictures/picture} % The first bracket is the picture height, the second is the thickness of the frame around the picture (0pt for no frame)
%\quote{}

%----------------------------------------------------------------------------------------

\begin{document}

\begin{wrapfigure}{r}{100pt}
\centering
\includegraphics[width=100pt]{yu} 
\label{fig:wrapfig}
\end{wrapfigure}

\hspace{150pt}
\vspace{10pt}
\textit{\Huge{\textcolor{black}{张宇}}}

\vspace{10pt} 

出生年月 \quad 1992.04

\vspace{5pt} 

电子邮箱 \quad zhangyumath@tju.edu.cn

\vspace{5pt} 

个人主页 \quad https://yuzhangmath.github.io/

\vspace{5pt} 

研究方向 \quad 基础数学: 代数拓扑、范畴理论

\vspace{5pt} 

工作地址 \quad 天津市南开区卫津路92号,天津大学应用数学中心,

\vspace{2pt} 

\hspace{60pt} 邮编300072



\vspace{10pt} 

\section{工作经历}

\vspace{3pt} 

\cvitem{2023.09 -- ~ 至今~ }{\textbf{讲师}~ ~  \quad 天津大学 \quad  \textit{天津, 中国} }

\vspace{3pt} 

\cvitem{2020.09 -- 2023.09}{\textbf{博士后} \quad  南开大学 \quad  \textit{天津, 中国} }

\vspace{5pt} 


\section{教育经历}

\vspace{3pt} 

\cvitem{2014.08 -- 2020.05}{\textbf{俄亥俄州立大学(The Ohio State University)} \quad \textit{哥伦布, 俄亥俄, 美国} 
\newline 数学博士,  导师: John E. Harper}

\vspace{3pt} 

\cvitem{2010.09 -- 2014.07}{\textbf{北京大学} \quad \textit{北京, 中国}
\newline 数学学士, 导师: 范辉军}

\vspace{5pt} 

%\vspace{2pt} 

%\cvitem{2007.09 -- 2010.06}{\textbf{唐山市第一中学} \textit{唐山, 河北, 中国}
%}


\section{科研基金}

\vspace{3pt} 

\cvitem{2025.01 -- 2027.12}{国家自然科学基金青年项目(No. 12401085):基于motivic理论与计算机辅助计算的球面稳定同伦群奇质数分量群研究;\textbf{项目负责人}。}

\vspace{3pt} 

\cvitem{2024.01 -- 2025.12}{天津大学科技创新领军人才培育计划启明项目(No. 2024XQM-0009):球面稳定同伦群的计算;\textbf{项目负责人}。}

\vspace{3pt} 

\cvitem{2023.01 -- 2026.12}{国家自然科学基金地区科学基金项目(No. 12261091):有关谱序列及分类空间$BPU_n$的上同调群的若干问题研究;\textbf{合作单位项目负责人}。}

\vspace{3pt} 

\cvitem{2023.01 -- 2026.12}{国家自然科学基金面上项目(No. 12271183):环面空间的上同调与motivic稳定同伦;\textbf{主要参与者}。}

\vspace{3pt} 

\cvitem{2021.05 -- 2023.04}{博士后国际交流计划引进项目: 幂零环结构谱的性质研究;\textbf{项目负责人}。}

\vspace{3pt} 

\cvitem{2019.01 -- 2022.12}{国家自然科学基金面上项目(No. 11871284): Motivic稳定同伦与环面拓扑中R-S谱序列的研究;\textbf{主要参与者}。}

\vspace{20pt} 



\section{学术论文}

\vspace{3pt} 

\begin{itemize}[wide=30pt, leftmargin=*]
    \item \textbf{The secondary periodic element $\beta_{p^2/p^2-1}$ and its applications}
        \newline (joint with J. Hong and X. Wang), \textit{SCIENCE CHINA Mathematics}. 68, 207-222 (2025).
    \item \textbf{Detecting nontrivial products in the stable homotopy ring of spheres via the third Morava stabilizer algebra}
    \newline (joint with X. Wang, J. Wu, and L. Zhong), \textit{Proceedings of the American Mathematical Society}. 152 (2024), 4521-4536.
    \item \textbf{A correspondence between higher Adams differentials and higher algebraic Novikov differentials at odd primes}
        \newline (joint with X. Wang), \textit{Proceedings of the American Mathematical Society}. 151 (2023), 5087-5096.
    \item \textbf{The $p$-primary subgroup of the cohomology of $BPU_n$ in dimension $2p+6$}
        \newline (joint with Z. Zhang and L. Zhong), \textit{Topology and Its Applications}.  338 (2023): 108642.
    \item \textbf{Some nontrivial secondary Adams differentials on the fourth line}
        \newline (joint with X. Wang and Y. Wang),  \textit{New York Journal of Mathematics}.  29 (2023) 687-707.
    \item \textbf{Homotopy pro-nilpotent structured ring spectra and topological Quillen localization}
        \newline \textit{Journal of Homotopy and Related Structures}. 17(4), 511-523 (2022).
    \item \textbf{The $p$-primary subgroups of the cohomology of $BPU_n$ in dimensions less than $2p+5$}
        \newline (joint with X. Gu, Z. Zhang, and L. Zhong), \textit{Proceedings of the American Mathematical Society}. 150(9), 4099-4111 (2022). 
    \item \textbf{Topological Quillen localization of structured ring spectra}
        \newline (joint with J. E. Harper), \textit{Tbilisi Mathematical Journal}. 12(3), 69-91 (2019). 
\end{itemize}

\vspace{2pt} 







\section{学术报告}

\vspace{2pt} 

\cvitem{2024年10月}{\textbf{延边大学}, 拓扑讨论班 (线上), "The convergence of $h_0h_3$ in the odd primary Adams spectral sequence"}

\vspace{2pt} 

\cvitem{2023年12月}{\textbf{北京大学}, 几何与拓扑讨论班, "The algebraic Novikov spectral sequence and stable homotopy groups of spheres at odd primes"}

\vspace{2pt} 

\cvitem{2023年05月}{\textbf{南开大学}, 2023年拓扑及其应用会议, “Motivic method and stable homotopy groups of spheres”}

\vspace{2pt} 

\cvitem{2022年11月}{\textbf{天津理工大学}, 天津市数学与统计学联合学术年会(线上), “Stable homotopy groups of the sphere at odd primes”}

\vspace{2pt} 

\cvitem{2022年08月}{\textbf{内蒙古民族大学}, 第八届全国青年拓扑学者论坛, “Some Differentials in the Algebraic Novikov Spectral Sequence and Its Applications”}

\vspace{2pt} 

\cvitem{2021年11月}{\textbf{陈省身数学研究所}, 几何与物理中的高阶结构(线上), “Structured ring spectra and Topological Quillen homology”}

\vspace{2pt} 

\cvitem{2021年07月}{\textbf{上海数学中心}, IWoAT 2021 Junior Researcher Forum, “Some spectral sequence computations towards stable homotopy groups of spheres at odd primes”}

\vspace{2pt} 

\cvitem{2021年07月}{\textbf{东北师范大学}, 第七届全国青年拓扑学者论坛, “Some spectral sequence computations towards stable homotopy groups of spheres at odd primes”}

\vspace{2pt} 

\cvitem{2020年12月}{\textbf{Vietnam National University}, The 8th East Asian Conference on Algebraic Topology (online), “Koszul duality and TQ-homological Whitehead theorem of structured ring spectra”}

\vspace{2pt} 

\cvitem{2020年11月}{\textbf{University of Regina}, Topology Seminar (online), “Koszul duality and TQ-homological Whitehead theorem of structured ring spectra”}

\vspace{2pt} 

\cvitem{2019年11月}{\textbf{Binghamton University}, Geometry and Topology Seminar, “Topological Quillen localization of structured ring spectra”}

\vspace{2pt} 

\cvitem{2019年11月}{\textbf{The University of Chicago}, Algebraic Topology Seminar, “Topological Quillen localization of structured ring spectra”}

\vspace{2pt} 

\cvitem{2019年11月}{\textbf{Penn State Altoona}, Topology Seminar, “Topological Quillen localization of structured ring spectra”}

\vspace{2pt} 

\cvitem{2019年10月}{\textbf{Indiana University Bloomington}, Topology Seminar, “Topological Quillen localization of structured ring spectra”}

\vspace{2pt} 

\cvitem{2019年10月}{\textbf{University of Illinois at Urbana-Champaign}, Topology Seminar, “Topological Quillen localization of structured ring spectra”}

\vspace{2pt} 

\cvitem{2019年07月}{\textbf{Norwegian University of Science and Technology}, Equivariant Topology and Derived Algebra conference, “Topological Quillen completion and localization of structured ring spectra”}

\vspace{2pt} 

\cvitem{2019年03月}{\textbf{University of Illinois at Urbana-Champaign}, Graduate Student Topology and Geometry Conference, “An easy proof of the homological Whitehead theorem for nilpotent spaces”}

\vspace{2pt} 

\cvitem{2018年09月}{\textbf{The Ohio State University}, Homotopy Theory Seminar, “Localization of structured ring spectra with respect to TQ homology”}

\vspace{2pt}

\cvitem{2018年07月}{\textbf{University of Copenhagen}, Young Topologists Meeting, “Bousfield localization of structured ring spectra”}

\vspace{2pt}

\section{组织会议}

\vspace{2pt} 

\cvitem{2023年05月}{\textbf{2023年拓扑及其应用会议}, 南开大学,组织委员会成员}

\section{出席会议}

\vspace{2pt} 

\cvitem{2024年09月}{\textbf{几何与拓扑中的计算机辅助研究}, 天元数学国际交流中心}

\vspace{2pt} 

\cvitem{2024年07月}{\textbf{International conference at Fudan University}, 上海数学中心, 复旦大学}

\vspace{2pt} 

\cvitem{2024年06月}{\textbf{IWoAT 2024 summer school on motivic homotopy theory}, 上海数学中心, 复旦大学}

\vspace{2pt} 

\cvitem{2023年12月}{\textbf{International Workshop on Algebraic Topology 2023 Winter School}, 南方科技大学}

\vspace{2pt} 

\cvitem{2023年08月}{\textbf{IWoAT Summer School 2023: Operads, spectra, and multiplicative structures}, 北京雁栖湖应用数学研究院}

\vspace{2pt} 

\cvitem{2023年07月}{\textbf{International Workshop on Algebraic Topology 2023}, 北京国际数学中心}

\vspace{2pt} 

\cvitem{2023年05月}{\textbf{2023年拓扑及其应用会议}, 南开大学}

\vspace{2pt} 

\cvitem{2022年11月}{\textbf{天津市数学与统计学联合学术年会(线上)}, 天津理工大学与南开大学联合举办}

\vspace{2pt}

\cvitem{2022年08月}{\textbf{Summer School on Chromatic Homotopy Theory and Higher (Infinity-Categorical) Algebra (online)}, 上海数学中心, 复旦大学, 与北京雁栖湖应用数学研究院联合举办}

\vspace{2pt}

\cvitem{2022年08月}{\textbf{第八届全国青年拓扑学者论坛}, 内蒙古民族大学}

\vspace{2pt}

\cvitem{2022年07月}{\textbf{Mid-South Algebraic Topology and Geometry Workshop (online)}, 华中科技大学与华南理工大学联合举办}

\vspace{2pt}

\cvitem{2021年11月}{\textbf{几何与物理中的高阶结构(线上)}, 陈省身数学研究所}

\vspace{2pt}

\cvitem{2021年07月}{\textbf{IWoAT 2021 Junior Researcher Forum (online)}, 上海数学中心}

\vspace{2pt}

\cvitem{2021年07月}{\textbf{Summer School on Equivariant Homotopy Theory}, 上海数学中心}

\vspace{2pt}

\cvitem{2021年07月}{\textbf{第七届全国青年拓扑学者论坛}, 东北师范大学}

\vspace{2pt}

\cvitem{2020年12月}{\textbf{The 8th East Asian Conference on Algebraic Topology (online)}, organized jointly by Vietnam National University and Vietnam Institute for Advanced Study in Mathematics}

\vspace{2pt}

\cvitem{2020年08月}{\textbf{第一届拓扑与数据科学联合工作营}, 重庆理工大学}

\vspace{2pt}

\cvitem{2020年05月}{\textbf{Midwest Topology Seminar (online)}, organized by Wayne State University}

\vspace{2pt}

\cvitem{2019年10月}{\textbf{Mayday 2019}, at The University of Chicago}

\vspace{2pt}

\cvitem{2019年09月}{\textbf{AMS Fall Central Sectional Meeting}, at University of Wisconsin-Madison}

\vspace{2pt}


\cvitem{2019年08月}{\textbf{International Workshop on Algebraic Topology}, 上海数学中心}

\vspace{2pt}


\cvitem{2019年08月}{\textbf{Summer School on Equivariant Homotopy Theory}, 上海数学中心}

\vspace{2pt}


\cvitem{2019年07月}{\textbf{Equivariant Topology \& Derived Algebra}, at Norwegian University of Science and Technology}

\vspace{2pt}


\cvitem{2019年07月}{\textbf{Young Topologists Meeting}, at École polytechnique fédérale de Lausanne}

\vspace{2pt}


\cvitem{2019年05月}{\textbf{Midwest Topology Seminar}, at Michigan State University}

\vspace{2pt}


\cvitem{2019年04月}{\textbf{Shanks Workshop on Homotopy Theory}, at Vanderbilt University}

\vspace{2pt}


\cvitem{2019年03月}{\textbf{Graduate Student Topology and Geometry Conference}, at University of Illinois at Urbana-Champaign}

\vspace{2pt}

\cvitem{2019年03月}{\textbf{Workshop on Functor Calculus}, at The Ohio State University}

\vspace{2pt}


\cvitem{2018年09月}{\textbf{Midwest Topology Seminar}, at University of Kentucky}

\vspace{2pt}

\cvitem{2018年07月}{\textbf{Young Topologists Meeting}, at University of Copenhagen}

\vspace{2pt}


\cvitem{2018年04月}{\textbf{Midwest Topology Seminar}, at Indiana University Bloomington}

\vspace{2pt}


\cvitem{2018年03月}{\textbf{AMS Special Session on Homotopy Theory},  at The Ohio State University}

\vspace{2pt}


\cvitem{2018年03月}{\textbf{Midwest Topology Seminar}, at Northwestern University}

\vspace{2pt}

\cvitem{2017年11月}{\textbf{Midwest Topology Seminar}, at Wayne State University}

\vspace{2pt}

\cvitem{2017年07月}{\textbf{Homotopy Theory: Tools and Applications}, at University of Illinois at Urbana-Champaign}

\vspace{2pt}

\cvitem{2017年05月}{\textbf{The MIT Talbot Workshop: Obstruction theory for structured ring spectra}, at Gooding Talbot house}

\vspace{2pt}

\cvitem{2017年03月}{\textbf{Shanks Workshop on Homotopy Theory}, at Vanderbilt University}

\vspace{2pt}

\cvitem{2017年02月}{\textbf{Conference for Young researchers in homotopy theory and categorical structures}, at Max Planck Institute for Mathematics}

\vspace{2pt}



\section{教学经历}

\vspace{2pt}
 
\cvitem{}{\textbf{在天津大学作为主讲老师授课:}}

\vspace{2pt}

\cvitem{2024年秋季}{微积分I \quad 学生162人,共96学时。
}

\vspace{5pt}
 
\cvitem{}{\textbf{在南开大学作为主讲老师授课:}}

\vspace{2pt}

\cvitem{2022年春季}{文科概率统计 \quad 学生141人,每周3课时。
\newline 学生给出的量化评教平均分为:97.7 (满分100)。}

\vspace{2pt}

\cvitem{2021年秋季}{文科高等数学 \quad 学生111人,每周4课时。
\newline 学生给出的量化评教平均分为:99.0 (满分100)。}

\vspace{5pt}
 
\cvitem{}{\textbf{在俄亥俄州立大学作为助教讲授习题课:}}

\vspace{2pt}

\cvitem{2018年秋季}{工科微积分3 \quad 三个教学班,合计每周3课时。} 

\vspace{2pt}

\cvitem{2017年秋季}{工科微积分2 \quad 两个教学班,合计每周4课时。}

\vspace{2pt}

\cvitem{2016年秋季}{工科微积分2 \quad 两个教学班,合计每周4课时。}

\vspace{2pt}

\cvitem{2016年春季}{微积分2 \quad  \quad  \quad 两个教学班,合计每周4课时。}

\vspace{2pt}



\section{掌握外语}

\vspace{2pt}

\hspace{22pt}
\textbf{英语}: 高水平听说读写 

\vspace{2pt}

\hspace{22pt}
\textbf{法语}: 基本阅读



\end{document}